
%NOTE: to be used with \usepackage{subfiles} in the main file.
%Subfiles go in folders which live with the main file.
%Bibliography and preamble go in the main file.

%%%%%%%%%%%%%%%%%%%%%%%% PREAMBLE %%%%%%%%%%%%%%%%%%%%%%%%
\providecommand{\main}{..}

\documentclass[main]{subfiles} %Each instance of `../' elevates one folder to find the main file

\begin{document}

%%%%%%%%%%%%%%%%%%%%%%% DOCUMENT %%%%%%%%%%%%%%%%%%%%%%%

% \tableofcontents % Can be useful to load a TOC while writing

\doublespacing

\schapter{Conclusions}
\vspace{20pt}

We can look at m-body subjettiness ($\tau_1^{(1)},\tau_1^{(0.5)},\tau_1^{(2)},\tau_2^{(1)},\tau_2^{(0.5)},\tau_2^{(2)}$).

A better study can be done by analysing the ideal parameters in the grooming techniques used before generating the samples. 

Following the hand rule for the radius, a better analysis could be done by studying the effect of decreasing the $R$ of the reconstructed fat jets as the $p_T$ of the boosted particles increases.

For simplicity the cut used in the $W$ analysis was the same as for the top analysis, a better study could use a lower $p_T$ cut for the $W$, because the Lorentz boost is achieved at lower $p_T$ values for this particle.

The performance of the tagging algorithm could be further studied by delving into the architecture of the different TMVA analyses, specially into those that seemed to work the best (BDT and DNN).




















% \bibliographystyle{../../PhilReview} %%bib style found in bst folder, in bibtex folder, in texmf folder.
% \nobibliography{Zotero} %%bib database found in bib folder, in bibtex folder
% \nobibliography{../../Thesis_bib}
\biblio

\end{document}
