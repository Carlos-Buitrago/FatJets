
%NOTE: to be used with \usepackage{subfiles} in the main file.
%Subfiles go in folders which live with the main file.
%Bibliography and preamble go in the main file.

%%%%%%%%%%%%%%%%%%%%%%%% PREAMBLE %%%%%%%%%%%%%%%%%%%%%%%%

\providecommand{\main}{..}

\documentclass[main]{subfiles} %Each instance of `../' elevates one folder to find the main file

\begin{document}

%%%%%%%%%%%%%%%%%%%%%%% DOCUMENT %%%%%%%%%%%%%%%%%%%%%%%

% \tableofcontents % Can be useful to load a TOC while writing

\doublespacing

\schapter{Introduction}
\vspace{20pt}

Despite its huge success, the Standard Model (SM) of particle physics is regarded as an incomplete theory, as has been shown multiple times in the past. Some of the challenges the SM faces include the hierarchy problem (which involves the mass of the Higgs boson), the absence of a candidate for dark matter and the matter-antimatter asymmetry of the universe. Several extensions to the SM have been proposed in order to deal with these issues and their predictions are constantly being put to test by the Large Hadron Collider (LHC) experiments. Most of these SM extensions involve models which predict the existence of new heavy particles with decay channels involving top quarks, electroweak bosons and Higgs bosons. If these new states are heavy enough their decay products are expected to have large transverse momenta greatly exceeding their rest masses ($p_T \gg m$). These kind of objects are called \textit{boosted objects}. \\ 

In light of the previous statements, the search for new physics at the LHC will continue to look further and further into previously unexplored kinematic regimes, and as a result of the high centre-of-mass energy that it has achieved, the LHC is already able to produce a large number of boosted particles across many final states. As the sensitivity of searches for new phenomena depends directly on them, it is of utmost importance to efficiently reconstruct and identify these kind of objects. In order to do so, many new techniques have been developed which rapidly gave birth to a new fast growing field of research. \\

When the boost factor is large enough, boosted objects decay into a highly collimated spray of hadrons. Because of this, the decay products of these kind of particles would be reconstructed as a single jet by standard jet algorithms. When boosted objects decay hadronically commonly used reconstruction methods become inefficient due to the large background of ordinary QCD jets originated from light quarks or gluons. \\ 

The identification of boosted hadronically decaying objects can be achieved by using jet substructure techniques. Looking at the decay products of boosted particles as collimated (and unresolved) jets separately most likely becomes a futile effort. Instead, a single large R jet (called a fat jet) containing all of the decay products can be reconstructed. The internal structure of these fat jets can be used to distinguish between those originating from boosted electroweak bosons and those originating from light quark or gluons. In order to achieve this, different jet shape methods are used, which take advantage of the difference in the energy patterns of signal and background jets. \\

This work will be mainly focused on the exploration of the "$N$-subjettiness" jet shape, denoted by $\tau_N$. This substructure variable gives us information about how much the radiation of a jet is aligned along $N$ different axes within it, effectively telling us how well a jet is described as having $N$ subjets. As will be discussed further below, the $\tau_N$ variables have discriminant capabilities which will be analysed in the present study. \\

Along with the jet mass, which also gives us information of the jet's substructure, $\tau_N$ will be used as discriminant variables in the tagging of boosted top quarks and boosted $W$ bosons. In order to carry out the tagging, these variables will be used as inputs for machine learning algorithms in the form of a multivariate analysis. The performance of the tagging method will be assessed, including a comparison between the different multivariate classifiers implemented and a comparison between the input variables proposed and other variables that are commonly used in the tagging of heavy particles. \\


% \hypsection{TITLE}


















% \bibliographystyle{../../PhilReview} %%bib style found in bst folder, in bibtex folder, in texmf folder.
% \nobibliography{Zotero} %%bib database found in bib folder, in bibtex folder
% \nobibliography{../../Thesis_bib}
\biblio

\end{document}
